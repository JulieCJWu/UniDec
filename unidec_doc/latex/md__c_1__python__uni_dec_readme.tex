Uni\+Dec is a Bayesian deconvolution program for deconvolution of mass spectra and ion mobility-\/mass spectra.

It was orignally published in\+: \href{http://pubs.acs.org/doi/abs/10.1021/acs.analchem.5b00140}{\tt M. T. Marty, A. J. Baldwin, E. G. Marklund, G. K. A. Hochberg, J. L. P. Benesch, C. V. Robinson, Anal. Chem. 2015, 87, 4370-\/4376.}

Detailed descriptions of the algorithm are provided in the paper. Please cite us if you use Uni\+Dec in your research.

Uni\+Dec may be downloaded from \href{http://unidec.chem.ox.ac.uk/}{\tt unidec.\+chem.\+ox.\+ac.\+uk}.

Please contact \href{mailto:michael.marty@chem.ox.ac.uk}{\tt michael.\+marty@chem.\+ox.\+ac.\+uk} for questions, suggestions, or with any bugs.

\subsection*{Installing}

\subsubsection*{Python}

There are several python libraries that Uni\+Dec will depend on.

matplotlib numpy scipy wxpython natsort twython pymzml networkx

With the exception of wxpython, which can be installed from the \href{http://wxpython.org/}{\tt web}, all of these can be installed from the command line with (for example)\+: \begin{DoxyVerb}pip install natsort
\end{DoxyVerb}


Note\+: I would highly recommend setting up 64-\/bit Python as the default. M\+S data works fine with 32-\/bit, but I\+M-\/\+M\+S data is prone to crash the memory. If you are getting memory errors, the first thing to try is to upgrade the bit level to 64.

\subsubsection*{Downloading the Binaries}

As described below, the Python code presented here relies on two critical binaries, Uni\+Dec.\+exe and Uni\+Dec\+I\+M.\+exe.

These binaries are available for download at \href{http://unidec.chem.ox.ac.uk/}{\tt unidec.\+chem.\+ox.\+ac.\+uk}.

These binaries should be deposited in the /unidec\+\_\+bin directory once downloaded.

If you want to convert Waters .Raw files, you will also need to add cdt.\+dll (for I\+M-\/\+M\+S) and Mass\+Lynx\+Raw.\+dll (for M\+S) to the same directory. These files can be found \href{http://www.waters.com/waters/supportList.htm?cid=511442&locale=en_GB&filter=documenttype|DWNL&locale=en_GB}{\tt here} with support numbers D\+W\+N\+L134825112 and D\+W\+N\+L134815627. Unfortunately, the libraries provided by Waters do not work on certain instruments, such as the G2-\/\+Si.

I have binaries built for Mac and Linux as well. They are a bit slower than the Windows version because they are compiled with gcc rather than the Intel C Compiler, but they are perfectly functional and still pretty darn fast. I can send these to you on request.

\subsection*{Uni\+Dec Documentation}

This documentation is for the Python engine and G\+U\+I.

My goal is that this documentation will allow you to utilize the power of the Uni\+Dec python engine for scripting data analysis routines and performing custom analysis. Also, it will allow you to add new modules to the Uni\+Dec G\+U\+I.

I\textquotesingle{}m still working on documenting some of the windows and extensions, but the core features should be here.

The C code is proprietary and not open source. The copyright is owned by the University of Oxford. If you are interested in the C source code, please contact me. It should be possible to share with appropriate licensing or research agreements.

\subsection*{Uni\+Dec Architecture}

Uni\+Dec is bilingual. The core of the algorithm is written in C and compiled as a binary. It can be run independently as a command line program fed by a configuration file.

The Python engine and G\+U\+I serve as a very extensive wrapper for the C core.

The engine (unidec.\+py) can be operated by independently of the G\+U\+I. This allows scripting of Uni\+Dec analysis for more complex and high-\/throughput analysis than is possible with the G\+U\+I. The engine contains three major subclasses, a config, data, and peaks object.

The G\+U\+I is organized with a Model-\/\+Presenter-\/\+View architecture. The main App is the presenter (G\+Uni\+Dec.\+py). The presenter contains a model (the Uni\+Dec engine at unidec.\+py) and a view (mainwindow.\+py). The presenter coordinates transfer between the G\+U\+I and the engine.

\subsection*{Getting Started}

Here is some sample code for how to use the engine. \begin{DoxyVerb}import unidec

file_name="test.txt"
folder="C:\\data"

eng=unidec.UniDec()

eng.open_file(file_name, folder)

eng.process_data()
eng.run_unidec(silent=True)
eng.pick_peaks()
\end{DoxyVerb}


In reading the documentation, it is perhaps best to start with the unidec.\+Uni\+Dec class. The main G\+U\+I class is G\+Uni\+Dec.\+Uni\+Dec\+App.

\subsection*{Licensing}

Uni\+Dec is free for noncommercial use under the \href{http://unidec.chem.ox.ac.uk/12116_UniDec_Academic%20Use%20Licence.pdf}{\tt Uni\+Dec Academic License}

Commercial licensing is available. Details are provided at the bottom of the Academic License.

The Python G\+U\+I and engine are licensed under the \href{http://www.gnu.org/licenses/gpl-3.0.en.html}{\tt G\+N\+U Public License (v.\+3)}. 